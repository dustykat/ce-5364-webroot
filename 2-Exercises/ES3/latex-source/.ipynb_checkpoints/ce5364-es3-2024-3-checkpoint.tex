\documentclass[12pt]{article}
\usepackage{geometry}                % See geometry.pdf to learn the layout options. There are lots.
\geometry{letterpaper}                   % ... or a4paper or a5paper or ... 
%\geometry{landscape}                % Activate for for rotated page geometry
\usepackage[parfill]{parskip}    % Activate to begin paragraphs with an empty line rather than an indent
\usepackage{daves,fancyhdr,natbib,graphicx,dcolumn,amsmath,lastpage,url}
\usepackage{amsmath,amssymb,epstopdf,longtable}
\usepackage[final]{pdfpages}
\DeclareGraphicsRule{.tif}{png}{.png}{`convert #1 `dirname #1`/`basename #1 .tif`.png}
\pagestyle{fancy}
\lhead{CE 5364 -- Groundwater Transport Phenomena }
\rhead{FALL 2024}
\lfoot{ES3}
\cfoot{}
\rfoot{Page \thepage\ of \pageref{LastPage}}
\renewcommand\headrulewidth{0pt}



\begin{document}
\begin{center}
{\textbf{{ CE 5364 Groundwater Transport Phenomena } \\ {Exercise Set 3}}}
\end{center}

\section*{\small{Exercises}}
\begin{enumerate} %% Problem Counter

%%%%%%%%%%%%%%%%%%%%%%%%%%%%%%%%%%%%%%%%%%%%%%%%%%%%

\item (Problem 6-7, pg. 588)

An instantaneous release of biodegradable constituients occurs in a 1-D aquifer. The mass released is $1.0 ~kg$ over a $10~m^2$ area normal to the flow direction, $\alpha_l = 1.0~m$, the seepage velocity is 1.0 $\frac{m}{day}$, and the half-life of the decaying constituient is 33 years.

Determine:
\begin{enumerate} %% Deliverable Counter
\item Sketch the system.
\item The maximum concentration at 100 meters from the source.
\item Plot a concentration history (annual intervals) for a 40 year period from release date for a location 100 meters from the source.
%\item The volumetric flow rate through the column.
\end{enumerate} %% Deliverable Counter

%%%%%%%%%%%%%%%%%%%%%%%%%%%%%%%%%%%%%%%%%%%%%%%%%%%%%%%%%%
\item (Problem 6-9, pg. 589)

Discharge from a point source introduced $10~kg$ of contaminant to an aquifer. The seepage velocity is $0.1~\frac{ft}{day}$ in the $+x$ direction.  The longitudinal and transverse dispersion coefficients are $D_x = 0.01 \frac{ft^2}{day}$, and $D_y = D_z = 0.001 \frac{ft^2}{day}$, respectively.

Determine:
\begin{enumerate} %% Deliverable Counter
\item Sketch the system.
\item The maximum concentration at $x=100~ft$ and $t=5~years$.
\item The concentration at $(x,y,z,t) = (200~ft,5~ft,2~ft,5~years)$
\end{enumerate} %% Deliverable Counter
%\clearpage
%%%%%%%%%%%%%%%%%%%%%%%%%%%%%%%%%%%%%%%%%%%%%%%%%%%%%%
%%%%%%%%%%%%%%%%%%%%%%%%%%%%%%%%%%%%%%%%%%%%%%%%%%%%%%
\item (Problem 6-10, pg. 589)

Using the Domenico and Schwartz (1998) planar source model (pg. 182) to a continuous source that has been leaking contaminant into an aquifer for 15 years.  The source had width $Y=6~m$ and depth $Z=6~m$. The source concentration is $10~\frac{mg}{l}$. The seepage velocity is $0.057~\frac{m}{day}$. The longitudinal, transverse, and vertical dispervities are $1~m$,$0.1~m$, and $0.01~m$ respectively.

Determine:
\begin{enumerate} %% Deliverable Counter
\item Sketch the system.
\item The contaminant concentration history at a location $x=200~m$ from the source using 1-year increments for 30 years.
\end{enumerate} %% Deliverable Counter

%%%%%%%%%%%%%%%%%%%%%%%%%%%%%%%%%%%%%%%%%%%%%%%%%%%%%%%%%%%%%%%%%%%

\item (Data Analysis)

A batch isotherm test was performed with several 1-L solutions of the chemical of interest and one soil type, 20 g in each solution container.  The initial and final solution concentrations are shown in Table \ref{tab:Isotherm}.  Fit the linear, Freundlich, and Langmuir isotherm equations to this data.  

% Requires the booktabs if the memoir class is not being used
\begin{table}[htbp]
\centering
\caption{Isotherm Observations}
\begin{tabular}{p{1.5in}p{1.5in}} % Column formatting, @{} suppresses leading/trailing space
Initial Concentration (mg/L) & Equilibrium Concentration (mg/L)\\
\hline
\hline
7.10&6.71\\
4.53&4.18\\
1.89&1.63\\
1.31&1.10\\
1.03&0.85\\
\hline
\end{tabular}
\label{tab:Isotherm}
\end{table}

Determine:
\begin{enumerate} %% Deliverable Counter
\item The Linear isotherm equation for these data (i.e. fit the isotherm model to the data), plot the isotherm and data
\item The Freundlich isotherm equation for these data, plot the isotherm and data
\item The Langmuir isotherm equation for these data, plot the isotherm and data
\item Which isotherm model produces the best fit for these data?
\end{enumerate} %% Deliverable Counter

Show calculations and identify all fitted parameter values.
\clearpage
%%%%%%%%%%%%%%%%%%%%%%%%%%%%%%%%%%%%%%%%%%%%%%%%%%%%%
%%%%%%%%%%%%%%%%%%%%%%%%%%%%%%%%%%%%%%%%%%%%%%%%%%%%%

\item (Data Analysis)

The following table (Table \ref{tab:ColumnData})has data from a column test with bromide (conservative) and chromium (sorbed).  
The porosity of the soil was 0.485, the bulk density was 1.85 g/cc, velocity was 0.244 cm/min, and the column was 25.4 cm long with a diameter of 2.54 cm.  

% Requires the booktabs if the memoir class is not being used
\begin{table}[h!]
\centering
\caption{Effluent Breakthrough Observations}
\begin{tabular}{p{1.5in}p{1.5in}p{1.5in}} % Column formatting, @{} suppresses leading/trailing space
Time (min) & Bromide $\frac{C}{Co}$ & Chromium $\frac{C}{Co}$ \\
\hline
\hline
0&0.000&0.000\\
15&0.000&0.000\\
30&0.005&0.000\\
45&0.003&0.000\\
60&0.013&0.000\\
75&0.075&0.000\\
90&0.137&0.000\\
105&0.530&0.000\\
120&0.841&0.000\\
135&1.000&0.000\\
150&1.000&0.000\\
165&1.000&0.009\\
180&1.000&0.186\\
195&1.000&0.595\\
210&1.000&0.791\\
225&1.000&0.875\\
240&1.000&0.913\\
255&1.000&0.946\\
270&1.000&0.946\\
285&1.000&1.000\\
300&1.000&1.000\\
315&1.000&1.000\\
330&1.000&1.000\\
345&1.000&1.000\\
360&1.000&1.000\\
\hline
\end{tabular}
\label{tab:ColumnData}
\end{table}

Determine:
\begin{enumerate} %% Deliverable Counter
    \item Sketch the system.
    \item The dispersivity in $cm$ 
   \item The retardation coefficient for $Cr$.
\end{enumerate} %% Deliverable Counter

%%%%%%%%%%%%%%%

\end{enumerate}%% Problem Counter

\end{document}  