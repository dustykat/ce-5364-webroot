\documentclass[12pt]{article}
\usepackage{geometry}                % See geometry.pdf to learn the layout options. There are lots.
\geometry{letterpaper}                   % ... or a4paper or a5paper or ... 
%\geometry{landscape}                % Activate for for rotated page geometry
\usepackage[parfill]{parskip}    % Activate to begin paragraphs with an empty line rather than an indent
\usepackage{daves,fancyhdr,natbib,graphicx,dcolumn,amsmath,lastpage,url}
\usepackage{amsmath,amssymb,epstopdf,longtable}
\usepackage[final]{pdfpages}
\DeclareGraphicsRule{.tif}{png}{.png}{`convert #1 `dirname #1`/`basename #1 .tif`.png}
\pagestyle{fancy}
\lhead{CE 5364 -- Groundwater Transport Phenomena }
\rhead{FALL 2024}
\lfoot{ES1}
\cfoot{}
\rfoot{Page \thepage\ of \pageref{LastPage}}
\renewcommand\headrulewidth{0pt}



\begin{document}
\begin{center}
{\textbf{{ CE 5364 Groundwater Transport Phenomena } \\ {Exercise Set 2}}}
\end{center}

\section*{\small{Exercises}}
\begin{enumerate}
\item (Problem 6-1, pg. 567)
Chloride ($Cl^{-}$) is injected as a continuous source into a 1-D column 50 centimeters long at a seepage velocity of $10^{-3}~\frac{cm}{s}$.  The effluent concentration measured at $t=1800~s$ from the start of the injection is $0.3$ of the initial concentration, and at $t=2700~s$ the effluent concentration is measured to be $0.4$ of the initial concentration.

Determine:
\begin{enumerate}
\item Sketch the system.
\item The longitudinal dispersivity.
\item The dispersion coefficient.
%\item The volumetric flow rate through the column.
\end{enumerate}

\item (Problem 6-2, pg. 567)
Chloride ($Cl^{-}$) is injected as a continuous source into a 1-D column.
The system has Darcy velocity of $5.18 \times 10^{-3}~\frac{in}{day}$, a porosity of $n=0.30$, and longitudinal dispersivity of $5 m$.

Determine:
\begin{enumerate}
\item Sketch the system.
\item The ratio $\frac{C}{C_0}$ at a location 0.3 meters from the injection location after 5 days of injection. 
\item The ratio $\frac{C}{C_0}$ at a location 0.3 meters from the injection location after 5 days of injection, if the dispersivity is 4 times larger ($20 m$).
\item Comment on the difference in results.
\end{enumerate}

\clearpage

\item (Problem 6-3, pg. 587)
The estimated mass from an instantaneous release of benzene is $107 \frac{kg}{m^2}$ of a 1-D aquifer system. The aquifer has a seepage velocity of $0.03 \frac{in}{day}$ and a longitudinal **dispersion coefficient** of $9 \times 10^{-4}\frac{m^2}{day}$

Determine:
\begin{enumerate}
\item Sketch the system.
\item Plot a concentration profile at $t = 1~\text{year}$ for $x = 0$ to $x = 50$ inches, in 1-inch increments.
\item Plot a concentration history at $x=v\times (1~\text{year})$ (this value stays constant) for $t = 0$ to $t = 2 $ years in $\frac{1}{12}$-year increments.
\item The maximum concentration at $t = 1~\text{year}$ and its location.
\end{enumerate}


\end{enumerate}
\end{document}  