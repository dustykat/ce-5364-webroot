\documentclass[11pt]{article}

    \usepackage[breakable]{tcolorbox}
    \usepackage{parskip} % Stop auto-indenting (to mimic markdown behaviour)
    

    % Basic figure setup, for now with no caption control since it's done
    % automatically by Pandoc (which extracts ![](path) syntax from Markdown).
    \usepackage{graphicx}
    % Keep aspect ratio if custom image width or height is specified
    \setkeys{Gin}{keepaspectratio}
    % Maintain compatibility with old templates. Remove in nbconvert 6.0
    \let\Oldincludegraphics\includegraphics
    % Ensure that by default, figures have no caption (until we provide a
    % proper Figure object with a Caption API and a way to capture that
    % in the conversion process - todo).
    \usepackage{caption}
    \DeclareCaptionFormat{nocaption}{}
    \captionsetup{format=nocaption,aboveskip=0pt,belowskip=0pt}

    \usepackage{float}
    \floatplacement{figure}{H} % forces figures to be placed at the correct location
    \usepackage{xcolor} % Allow colors to be defined
    \usepackage{enumerate} % Needed for markdown enumerations to work
    \usepackage{geometry} % Used to adjust the document margins
    \usepackage{amsmath} % Equations
    \usepackage{amssymb} % Equations
    \usepackage{textcomp} % defines textquotesingle
    % Hack from http://tex.stackexchange.com/a/47451/13684:
    \AtBeginDocument{%
        \def\PYZsq{\textquotesingle}% Upright quotes in Pygmentized code
    }
    \usepackage{upquote} % Upright quotes for verbatim code
    \usepackage{eurosym} % defines \euro

    \usepackage{iftex}
    \ifPDFTeX
        \usepackage[T1]{fontenc}
        \IfFileExists{alphabeta.sty}{
              \usepackage{alphabeta}
          }{
              \usepackage[mathletters]{ucs}
              \usepackage[utf8x]{inputenc}
          }
    \else
        \usepackage{fontspec}
        \usepackage{unicode-math}
    \fi

    \usepackage{fancyvrb} % verbatim replacement that allows latex
    \usepackage{grffile} % extends the file name processing of package graphics
                         % to support a larger range
    \makeatletter % fix for old versions of grffile with XeLaTeX
    \@ifpackagelater{grffile}{2019/11/01}
    {
      % Do nothing on new versions
    }
    {
      \def\Gread@@xetex#1{%
        \IfFileExists{"\Gin@base".bb}%
        {\Gread@eps{\Gin@base.bb}}%
        {\Gread@@xetex@aux#1}%
      }
    }
    \makeatother
    \usepackage[Export]{adjustbox} % Used to constrain images to a maximum size
    \adjustboxset{max size={0.9\linewidth}{0.9\paperheight}}

    % The hyperref package gives us a pdf with properly built
    % internal navigation ('pdf bookmarks' for the table of contents,
    % internal cross-reference links, web links for URLs, etc.)
    \usepackage{hyperref}
    % The default LaTeX title has an obnoxious amount of whitespace. By default,
    % titling removes some of it. It also provides customization options.
    \usepackage{titling}
    \usepackage{longtable} % longtable support required by pandoc >1.10
    \usepackage{booktabs}  % table support for pandoc > 1.12.2
    \usepackage{array}     % table support for pandoc >= 2.11.3
    \usepackage{calc}      % table minipage width calculation for pandoc >= 2.11.1
    \usepackage[inline]{enumitem} % IRkernel/repr support (it uses the enumerate* environment)
    \usepackage[normalem]{ulem} % ulem is needed to support strikethroughs (\sout)
                                % normalem makes italics be italics, not underlines
    \usepackage{soul}      % strikethrough (\st) support for pandoc >= 3.0.0
    \usepackage{mathrsfs}
    

    
    % Colors for the hyperref package
    \definecolor{urlcolor}{rgb}{0,.145,.698}
    \definecolor{linkcolor}{rgb}{.71,0.21,0.01}
    \definecolor{citecolor}{rgb}{.12,.54,.11}

    % ANSI colors
    \definecolor{ansi-black}{HTML}{3E424D}
    \definecolor{ansi-black-intense}{HTML}{282C36}
    \definecolor{ansi-red}{HTML}{E75C58}
    \definecolor{ansi-red-intense}{HTML}{B22B31}
    \definecolor{ansi-green}{HTML}{00A250}
    \definecolor{ansi-green-intense}{HTML}{007427}
    \definecolor{ansi-yellow}{HTML}{DDB62B}
    \definecolor{ansi-yellow-intense}{HTML}{B27D12}
    \definecolor{ansi-blue}{HTML}{208FFB}
    \definecolor{ansi-blue-intense}{HTML}{0065CA}
    \definecolor{ansi-magenta}{HTML}{D160C4}
    \definecolor{ansi-magenta-intense}{HTML}{A03196}
    \definecolor{ansi-cyan}{HTML}{60C6C8}
    \definecolor{ansi-cyan-intense}{HTML}{258F8F}
    \definecolor{ansi-white}{HTML}{C5C1B4}
    \definecolor{ansi-white-intense}{HTML}{A1A6B2}
    \definecolor{ansi-default-inverse-fg}{HTML}{FFFFFF}
    \definecolor{ansi-default-inverse-bg}{HTML}{000000}

    % common color for the border for error outputs.
    \definecolor{outerrorbackground}{HTML}{FFDFDF}

    % commands and environments needed by pandoc snippets
    % extracted from the output of `pandoc -s`
    \providecommand{\tightlist}{%
      \setlength{\itemsep}{0pt}\setlength{\parskip}{0pt}}
    \DefineVerbatimEnvironment{Highlighting}{Verbatim}{commandchars=\\\{\}}
    % Add ',fontsize=\small' for more characters per line
    \newenvironment{Shaded}{}{}
    \newcommand{\KeywordTok}[1]{\textcolor[rgb]{0.00,0.44,0.13}{\textbf{{#1}}}}
    \newcommand{\DataTypeTok}[1]{\textcolor[rgb]{0.56,0.13,0.00}{{#1}}}
    \newcommand{\DecValTok}[1]{\textcolor[rgb]{0.25,0.63,0.44}{{#1}}}
    \newcommand{\BaseNTok}[1]{\textcolor[rgb]{0.25,0.63,0.44}{{#1}}}
    \newcommand{\FloatTok}[1]{\textcolor[rgb]{0.25,0.63,0.44}{{#1}}}
    \newcommand{\CharTok}[1]{\textcolor[rgb]{0.25,0.44,0.63}{{#1}}}
    \newcommand{\StringTok}[1]{\textcolor[rgb]{0.25,0.44,0.63}{{#1}}}
    \newcommand{\CommentTok}[1]{\textcolor[rgb]{0.38,0.63,0.69}{\textit{{#1}}}}
    \newcommand{\OtherTok}[1]{\textcolor[rgb]{0.00,0.44,0.13}{{#1}}}
    \newcommand{\AlertTok}[1]{\textcolor[rgb]{1.00,0.00,0.00}{\textbf{{#1}}}}
    \newcommand{\FunctionTok}[1]{\textcolor[rgb]{0.02,0.16,0.49}{{#1}}}
    \newcommand{\RegionMarkerTok}[1]{{#1}}
    \newcommand{\ErrorTok}[1]{\textcolor[rgb]{1.00,0.00,0.00}{\textbf{{#1}}}}
    \newcommand{\NormalTok}[1]{{#1}}

    % Additional commands for more recent versions of Pandoc
    \newcommand{\ConstantTok}[1]{\textcolor[rgb]{0.53,0.00,0.00}{{#1}}}
    \newcommand{\SpecialCharTok}[1]{\textcolor[rgb]{0.25,0.44,0.63}{{#1}}}
    \newcommand{\VerbatimStringTok}[1]{\textcolor[rgb]{0.25,0.44,0.63}{{#1}}}
    \newcommand{\SpecialStringTok}[1]{\textcolor[rgb]{0.73,0.40,0.53}{{#1}}}
    \newcommand{\ImportTok}[1]{{#1}}
    \newcommand{\DocumentationTok}[1]{\textcolor[rgb]{0.73,0.13,0.13}{\textit{{#1}}}}
    \newcommand{\AnnotationTok}[1]{\textcolor[rgb]{0.38,0.63,0.69}{\textbf{\textit{{#1}}}}}
    \newcommand{\CommentVarTok}[1]{\textcolor[rgb]{0.38,0.63,0.69}{\textbf{\textit{{#1}}}}}
    \newcommand{\VariableTok}[1]{\textcolor[rgb]{0.10,0.09,0.49}{{#1}}}
    \newcommand{\ControlFlowTok}[1]{\textcolor[rgb]{0.00,0.44,0.13}{\textbf{{#1}}}}
    \newcommand{\OperatorTok}[1]{\textcolor[rgb]{0.40,0.40,0.40}{{#1}}}
    \newcommand{\BuiltInTok}[1]{{#1}}
    \newcommand{\ExtensionTok}[1]{{#1}}
    \newcommand{\PreprocessorTok}[1]{\textcolor[rgb]{0.74,0.48,0.00}{{#1}}}
    \newcommand{\AttributeTok}[1]{\textcolor[rgb]{0.49,0.56,0.16}{{#1}}}
    \newcommand{\InformationTok}[1]{\textcolor[rgb]{0.38,0.63,0.69}{\textbf{\textit{{#1}}}}}
    \newcommand{\WarningTok}[1]{\textcolor[rgb]{0.38,0.63,0.69}{\textbf{\textit{{#1}}}}}


    % Define a nice break command that doesn't care if a line doesn't already
    % exist.
    \def\br{\hspace*{\fill} \\* }
    % Math Jax compatibility definitions
    \def\gt{>}
    \def\lt{<}
    \let\Oldtex\TeX
    \let\Oldlatex\LaTeX
    \renewcommand{\TeX}{\textrm{\Oldtex}}
    \renewcommand{\LaTeX}{\textrm{\Oldlatex}}
    % Document parameters
    % Document title
    \title{ES6-WS}
    
    
    
    
    
    
    
% Pygments definitions
\makeatletter
\def\PY@reset{\let\PY@it=\relax \let\PY@bf=\relax%
    \let\PY@ul=\relax \let\PY@tc=\relax%
    \let\PY@bc=\relax \let\PY@ff=\relax}
\def\PY@tok#1{\csname PY@tok@#1\endcsname}
\def\PY@toks#1+{\ifx\relax#1\empty\else%
    \PY@tok{#1}\expandafter\PY@toks\fi}
\def\PY@do#1{\PY@bc{\PY@tc{\PY@ul{%
    \PY@it{\PY@bf{\PY@ff{#1}}}}}}}
\def\PY#1#2{\PY@reset\PY@toks#1+\relax+\PY@do{#2}}

\@namedef{PY@tok@w}{\def\PY@tc##1{\textcolor[rgb]{0.73,0.73,0.73}{##1}}}
\@namedef{PY@tok@c}{\let\PY@it=\textit\def\PY@tc##1{\textcolor[rgb]{0.24,0.48,0.48}{##1}}}
\@namedef{PY@tok@cp}{\def\PY@tc##1{\textcolor[rgb]{0.61,0.40,0.00}{##1}}}
\@namedef{PY@tok@k}{\let\PY@bf=\textbf\def\PY@tc##1{\textcolor[rgb]{0.00,0.50,0.00}{##1}}}
\@namedef{PY@tok@kp}{\def\PY@tc##1{\textcolor[rgb]{0.00,0.50,0.00}{##1}}}
\@namedef{PY@tok@kt}{\def\PY@tc##1{\textcolor[rgb]{0.69,0.00,0.25}{##1}}}
\@namedef{PY@tok@o}{\def\PY@tc##1{\textcolor[rgb]{0.40,0.40,0.40}{##1}}}
\@namedef{PY@tok@ow}{\let\PY@bf=\textbf\def\PY@tc##1{\textcolor[rgb]{0.67,0.13,1.00}{##1}}}
\@namedef{PY@tok@nb}{\def\PY@tc##1{\textcolor[rgb]{0.00,0.50,0.00}{##1}}}
\@namedef{PY@tok@nf}{\def\PY@tc##1{\textcolor[rgb]{0.00,0.00,1.00}{##1}}}
\@namedef{PY@tok@nc}{\let\PY@bf=\textbf\def\PY@tc##1{\textcolor[rgb]{0.00,0.00,1.00}{##1}}}
\@namedef{PY@tok@nn}{\let\PY@bf=\textbf\def\PY@tc##1{\textcolor[rgb]{0.00,0.00,1.00}{##1}}}
\@namedef{PY@tok@ne}{\let\PY@bf=\textbf\def\PY@tc##1{\textcolor[rgb]{0.80,0.25,0.22}{##1}}}
\@namedef{PY@tok@nv}{\def\PY@tc##1{\textcolor[rgb]{0.10,0.09,0.49}{##1}}}
\@namedef{PY@tok@no}{\def\PY@tc##1{\textcolor[rgb]{0.53,0.00,0.00}{##1}}}
\@namedef{PY@tok@nl}{\def\PY@tc##1{\textcolor[rgb]{0.46,0.46,0.00}{##1}}}
\@namedef{PY@tok@ni}{\let\PY@bf=\textbf\def\PY@tc##1{\textcolor[rgb]{0.44,0.44,0.44}{##1}}}
\@namedef{PY@tok@na}{\def\PY@tc##1{\textcolor[rgb]{0.41,0.47,0.13}{##1}}}
\@namedef{PY@tok@nt}{\let\PY@bf=\textbf\def\PY@tc##1{\textcolor[rgb]{0.00,0.50,0.00}{##1}}}
\@namedef{PY@tok@nd}{\def\PY@tc##1{\textcolor[rgb]{0.67,0.13,1.00}{##1}}}
\@namedef{PY@tok@s}{\def\PY@tc##1{\textcolor[rgb]{0.73,0.13,0.13}{##1}}}
\@namedef{PY@tok@sd}{\let\PY@it=\textit\def\PY@tc##1{\textcolor[rgb]{0.73,0.13,0.13}{##1}}}
\@namedef{PY@tok@si}{\let\PY@bf=\textbf\def\PY@tc##1{\textcolor[rgb]{0.64,0.35,0.47}{##1}}}
\@namedef{PY@tok@se}{\let\PY@bf=\textbf\def\PY@tc##1{\textcolor[rgb]{0.67,0.36,0.12}{##1}}}
\@namedef{PY@tok@sr}{\def\PY@tc##1{\textcolor[rgb]{0.64,0.35,0.47}{##1}}}
\@namedef{PY@tok@ss}{\def\PY@tc##1{\textcolor[rgb]{0.10,0.09,0.49}{##1}}}
\@namedef{PY@tok@sx}{\def\PY@tc##1{\textcolor[rgb]{0.00,0.50,0.00}{##1}}}
\@namedef{PY@tok@m}{\def\PY@tc##1{\textcolor[rgb]{0.40,0.40,0.40}{##1}}}
\@namedef{PY@tok@gh}{\let\PY@bf=\textbf\def\PY@tc##1{\textcolor[rgb]{0.00,0.00,0.50}{##1}}}
\@namedef{PY@tok@gu}{\let\PY@bf=\textbf\def\PY@tc##1{\textcolor[rgb]{0.50,0.00,0.50}{##1}}}
\@namedef{PY@tok@gd}{\def\PY@tc##1{\textcolor[rgb]{0.63,0.00,0.00}{##1}}}
\@namedef{PY@tok@gi}{\def\PY@tc##1{\textcolor[rgb]{0.00,0.52,0.00}{##1}}}
\@namedef{PY@tok@gr}{\def\PY@tc##1{\textcolor[rgb]{0.89,0.00,0.00}{##1}}}
\@namedef{PY@tok@ge}{\let\PY@it=\textit}
\@namedef{PY@tok@gs}{\let\PY@bf=\textbf}
\@namedef{PY@tok@ges}{\let\PY@bf=\textbf\let\PY@it=\textit}
\@namedef{PY@tok@gp}{\let\PY@bf=\textbf\def\PY@tc##1{\textcolor[rgb]{0.00,0.00,0.50}{##1}}}
\@namedef{PY@tok@go}{\def\PY@tc##1{\textcolor[rgb]{0.44,0.44,0.44}{##1}}}
\@namedef{PY@tok@gt}{\def\PY@tc##1{\textcolor[rgb]{0.00,0.27,0.87}{##1}}}
\@namedef{PY@tok@err}{\def\PY@bc##1{{\setlength{\fboxsep}{\string -\fboxrule}\fcolorbox[rgb]{1.00,0.00,0.00}{1,1,1}{\strut ##1}}}}
\@namedef{PY@tok@kc}{\let\PY@bf=\textbf\def\PY@tc##1{\textcolor[rgb]{0.00,0.50,0.00}{##1}}}
\@namedef{PY@tok@kd}{\let\PY@bf=\textbf\def\PY@tc##1{\textcolor[rgb]{0.00,0.50,0.00}{##1}}}
\@namedef{PY@tok@kn}{\let\PY@bf=\textbf\def\PY@tc##1{\textcolor[rgb]{0.00,0.50,0.00}{##1}}}
\@namedef{PY@tok@kr}{\let\PY@bf=\textbf\def\PY@tc##1{\textcolor[rgb]{0.00,0.50,0.00}{##1}}}
\@namedef{PY@tok@bp}{\def\PY@tc##1{\textcolor[rgb]{0.00,0.50,0.00}{##1}}}
\@namedef{PY@tok@fm}{\def\PY@tc##1{\textcolor[rgb]{0.00,0.00,1.00}{##1}}}
\@namedef{PY@tok@vc}{\def\PY@tc##1{\textcolor[rgb]{0.10,0.09,0.49}{##1}}}
\@namedef{PY@tok@vg}{\def\PY@tc##1{\textcolor[rgb]{0.10,0.09,0.49}{##1}}}
\@namedef{PY@tok@vi}{\def\PY@tc##1{\textcolor[rgb]{0.10,0.09,0.49}{##1}}}
\@namedef{PY@tok@vm}{\def\PY@tc##1{\textcolor[rgb]{0.10,0.09,0.49}{##1}}}
\@namedef{PY@tok@sa}{\def\PY@tc##1{\textcolor[rgb]{0.73,0.13,0.13}{##1}}}
\@namedef{PY@tok@sb}{\def\PY@tc##1{\textcolor[rgb]{0.73,0.13,0.13}{##1}}}
\@namedef{PY@tok@sc}{\def\PY@tc##1{\textcolor[rgb]{0.73,0.13,0.13}{##1}}}
\@namedef{PY@tok@dl}{\def\PY@tc##1{\textcolor[rgb]{0.73,0.13,0.13}{##1}}}
\@namedef{PY@tok@s2}{\def\PY@tc##1{\textcolor[rgb]{0.73,0.13,0.13}{##1}}}
\@namedef{PY@tok@sh}{\def\PY@tc##1{\textcolor[rgb]{0.73,0.13,0.13}{##1}}}
\@namedef{PY@tok@s1}{\def\PY@tc##1{\textcolor[rgb]{0.73,0.13,0.13}{##1}}}
\@namedef{PY@tok@mb}{\def\PY@tc##1{\textcolor[rgb]{0.40,0.40,0.40}{##1}}}
\@namedef{PY@tok@mf}{\def\PY@tc##1{\textcolor[rgb]{0.40,0.40,0.40}{##1}}}
\@namedef{PY@tok@mh}{\def\PY@tc##1{\textcolor[rgb]{0.40,0.40,0.40}{##1}}}
\@namedef{PY@tok@mi}{\def\PY@tc##1{\textcolor[rgb]{0.40,0.40,0.40}{##1}}}
\@namedef{PY@tok@il}{\def\PY@tc##1{\textcolor[rgb]{0.40,0.40,0.40}{##1}}}
\@namedef{PY@tok@mo}{\def\PY@tc##1{\textcolor[rgb]{0.40,0.40,0.40}{##1}}}
\@namedef{PY@tok@ch}{\let\PY@it=\textit\def\PY@tc##1{\textcolor[rgb]{0.24,0.48,0.48}{##1}}}
\@namedef{PY@tok@cm}{\let\PY@it=\textit\def\PY@tc##1{\textcolor[rgb]{0.24,0.48,0.48}{##1}}}
\@namedef{PY@tok@cpf}{\let\PY@it=\textit\def\PY@tc##1{\textcolor[rgb]{0.24,0.48,0.48}{##1}}}
\@namedef{PY@tok@c1}{\let\PY@it=\textit\def\PY@tc##1{\textcolor[rgb]{0.24,0.48,0.48}{##1}}}
\@namedef{PY@tok@cs}{\let\PY@it=\textit\def\PY@tc##1{\textcolor[rgb]{0.24,0.48,0.48}{##1}}}

\def\PYZbs{\char`\\}
\def\PYZus{\char`\_}
\def\PYZob{\char`\{}
\def\PYZcb{\char`\}}
\def\PYZca{\char`\^}
\def\PYZam{\char`\&}
\def\PYZlt{\char`\<}
\def\PYZgt{\char`\>}
\def\PYZsh{\char`\#}
\def\PYZpc{\char`\%}
\def\PYZdl{\char`\$}
\def\PYZhy{\char`\-}
\def\PYZsq{\char`\'}
\def\PYZdq{\char`\"}
\def\PYZti{\char`\~}
% for compatibility with earlier versions
\def\PYZat{@}
\def\PYZlb{[}
\def\PYZrb{]}
\makeatother


    % For linebreaks inside Verbatim environment from package fancyvrb.
    \makeatletter
        \newbox\Wrappedcontinuationbox
        \newbox\Wrappedvisiblespacebox
        \newcommand*\Wrappedvisiblespace {\textcolor{red}{\textvisiblespace}}
        \newcommand*\Wrappedcontinuationsymbol {\textcolor{red}{\llap{\tiny$\m@th\hookrightarrow$}}}
        \newcommand*\Wrappedcontinuationindent {3ex }
        \newcommand*\Wrappedafterbreak {\kern\Wrappedcontinuationindent\copy\Wrappedcontinuationbox}
        % Take advantage of the already applied Pygments mark-up to insert
        % potential linebreaks for TeX processing.
        %        {, <, #, %, $, ' and ": go to next line.
        %        _, }, ^, &, >, - and ~: stay at end of broken line.
        % Use of \textquotesingle for straight quote.
        \newcommand*\Wrappedbreaksatspecials {%
            \def\PYGZus{\discretionary{\char`\_}{\Wrappedafterbreak}{\char`\_}}%
            \def\PYGZob{\discretionary{}{\Wrappedafterbreak\char`\{}{\char`\{}}%
            \def\PYGZcb{\discretionary{\char`\}}{\Wrappedafterbreak}{\char`\}}}%
            \def\PYGZca{\discretionary{\char`\^}{\Wrappedafterbreak}{\char`\^}}%
            \def\PYGZam{\discretionary{\char`\&}{\Wrappedafterbreak}{\char`\&}}%
            \def\PYGZlt{\discretionary{}{\Wrappedafterbreak\char`\<}{\char`\<}}%
            \def\PYGZgt{\discretionary{\char`\>}{\Wrappedafterbreak}{\char`\>}}%
            \def\PYGZsh{\discretionary{}{\Wrappedafterbreak\char`\#}{\char`\#}}%
            \def\PYGZpc{\discretionary{}{\Wrappedafterbreak\char`\%}{\char`\%}}%
            \def\PYGZdl{\discretionary{}{\Wrappedafterbreak\char`\$}{\char`\$}}%
            \def\PYGZhy{\discretionary{\char`\-}{\Wrappedafterbreak}{\char`\-}}%
            \def\PYGZsq{\discretionary{}{\Wrappedafterbreak\textquotesingle}{\textquotesingle}}%
            \def\PYGZdq{\discretionary{}{\Wrappedafterbreak\char`\"}{\char`\"}}%
            \def\PYGZti{\discretionary{\char`\~}{\Wrappedafterbreak}{\char`\~}}%
        }
        % Some characters . , ; ? ! / are not pygmentized.
        % This macro makes them "active" and they will insert potential linebreaks
        \newcommand*\Wrappedbreaksatpunct {%
            \lccode`\~`\.\lowercase{\def~}{\discretionary{\hbox{\char`\.}}{\Wrappedafterbreak}{\hbox{\char`\.}}}%
            \lccode`\~`\,\lowercase{\def~}{\discretionary{\hbox{\char`\,}}{\Wrappedafterbreak}{\hbox{\char`\,}}}%
            \lccode`\~`\;\lowercase{\def~}{\discretionary{\hbox{\char`\;}}{\Wrappedafterbreak}{\hbox{\char`\;}}}%
            \lccode`\~`\:\lowercase{\def~}{\discretionary{\hbox{\char`\:}}{\Wrappedafterbreak}{\hbox{\char`\:}}}%
            \lccode`\~`\?\lowercase{\def~}{\discretionary{\hbox{\char`\?}}{\Wrappedafterbreak}{\hbox{\char`\?}}}%
            \lccode`\~`\!\lowercase{\def~}{\discretionary{\hbox{\char`\!}}{\Wrappedafterbreak}{\hbox{\char`\!}}}%
            \lccode`\~`\/\lowercase{\def~}{\discretionary{\hbox{\char`\/}}{\Wrappedafterbreak}{\hbox{\char`\/}}}%
            \catcode`\.\active
            \catcode`\,\active
            \catcode`\;\active
            \catcode`\:\active
            \catcode`\?\active
            \catcode`\!\active
            \catcode`\/\active
            \lccode`\~`\~
        }
    \makeatother

    \let\OriginalVerbatim=\Verbatim
    \makeatletter
    \renewcommand{\Verbatim}[1][1]{%
        %\parskip\z@skip
        \sbox\Wrappedcontinuationbox {\Wrappedcontinuationsymbol}%
        \sbox\Wrappedvisiblespacebox {\FV@SetupFont\Wrappedvisiblespace}%
        \def\FancyVerbFormatLine ##1{\hsize\linewidth
            \vtop{\raggedright\hyphenpenalty\z@\exhyphenpenalty\z@
                \doublehyphendemerits\z@\finalhyphendemerits\z@
                \strut ##1\strut}%
        }%
        % If the linebreak is at a space, the latter will be displayed as visible
        % space at end of first line, and a continuation symbol starts next line.
        % Stretch/shrink are however usually zero for typewriter font.
        \def\FV@Space {%
            \nobreak\hskip\z@ plus\fontdimen3\font minus\fontdimen4\font
            \discretionary{\copy\Wrappedvisiblespacebox}{\Wrappedafterbreak}
            {\kern\fontdimen2\font}%
        }%

        % Allow breaks at special characters using \PYG... macros.
        \Wrappedbreaksatspecials
        % Breaks at punctuation characters . , ; ? ! and / need catcode=\active
        \OriginalVerbatim[#1,codes*=\Wrappedbreaksatpunct]%
    }
    \makeatother

    % Exact colors from NB
    \definecolor{incolor}{HTML}{303F9F}
    \definecolor{outcolor}{HTML}{D84315}
    \definecolor{cellborder}{HTML}{CFCFCF}
    \definecolor{cellbackground}{HTML}{F7F7F7}

    % prompt
    \makeatletter
    \newcommand{\boxspacing}{\kern\kvtcb@left@rule\kern\kvtcb@boxsep}
    \makeatother
    \newcommand{\prompt}[4]{
        {\ttfamily\llap{{\color{#2}[#3]:\hspace{3pt}#4}}\vspace{-\baselineskip}}
    }
    

    
    % Prevent overflowing lines due to hard-to-break entities
    \sloppy
    % Setup hyperref package
    \hypersetup{
      breaklinks=true,  % so long urls are correctly broken across lines
      colorlinks=true,
      urlcolor=urlcolor,
      linkcolor=linkcolor,
      citecolor=citecolor,
      }
    % Slightly bigger margins than the latex defaults
    
    \geometry{verbose,tmargin=1in,bmargin=1in,lmargin=1in,rmargin=1in}
    
    

\begin{document}
    
    \maketitle
    
    

    
    \hypertarget{ce-5364-groundwater-transport-phenemona-fall-2023-exercise-set-6}{%
\section{\texorpdfstring{CE 5364 Groundwater Transport Phenemona Fall
2023 Exercise Set
6}{CE 5364 Groundwater Transport Phenemona   Fall 2023 Exercise Set 6}}\label{ce-5364-groundwater-transport-phenemona-fall-2023-exercise-set-6}}

\textbf{LAST NAME, FIRST NAME}

\textbf{R00000000}

\hypertarget{purpose}{%
\subsubsection{Purpose :}\label{purpose}}

Apply selected risk assessment methods

\hypertarget{assessment-criteria}{%
\subsubsection{Assessment Criteria :}\label{assessment-criteria}}

Completion, results plausible, format correct, example calculations
shown.

    \hypertarget{problem-1}{%
\subsection{Problem 1}\label{problem-1}}

Improper waste disposal practices at an industrial site resulted in
contamination of the soil on site by cadmium, a known carcinogen with a
slope factor of 6.10 \((\frac{mg}{kg d})^{-1}\). We will consider the
risk to off-site residents due to inhalation of airborne soil particles
that include the cadmium. Based on monitoring data, the concentration of
cadmium in the air off site is \(5.4 \times 10^{-4}~\frac{mg}{m^3}\).

Determine:

\begin{enumerate}
\def\labelenumi{\arabic{enumi}.}
\tightlist
\item
  CInh for residents that are children 1-6 years of age and adults.
\item
  The cancer risk due to these CInh values for the children and adults.
\end{enumerate}

Show all calculations and identify all parameter values used.

    \begin{tcolorbox}[breakable, size=fbox, boxrule=1pt, pad at break*=1mm,colback=cellbackground, colframe=cellborder]
\prompt{In}{incolor}{9}{\boxspacing}
\begin{Verbatim}[commandchars=\\\{\}]
\PY{c+c1}{\PYZsh{} Enter your solution below, or attach separate sheet(s) with your solution.}
\PY{c+c1}{\PYZsh{} Given}
\PY{n}{CA}\PY{o}{=}\PY{l+m+mf}{5.4e\PYZhy{}04} \PY{c+c1}{\PYZsh{} mg/m\PYZca{}3}
\PY{n}{SF}\PY{o}{=}\PY{l+m+mf}{6.10} \PY{c+c1}{\PYZsh{}(mg/kgd)\PYZca{}\PYZhy{}1}
\PY{c+c1}{\PYZsh{} use formula for CInh from readings}

\PY{k}{def} \PY{n+nf}{CInh}\PY{p}{(}\PY{n}{CA}\PY{p}{,}\PY{n}{IR}\PY{p}{,}\PY{n}{RR}\PY{p}{,}\PY{n}{ABSs}\PY{p}{,}\PY{n}{ET}\PY{p}{,}\PY{n}{EF}\PY{p}{,}\PY{n}{ED}\PY{p}{,}\PY{n}{BW}\PY{p}{,}\PY{n}{AT}\PY{p}{)}\PY{p}{:}
    \PY{n}{CInh} \PY{o}{=} \PY{p}{(}\PY{n}{CA}\PY{o}{*}\PY{n}{IR}\PY{o}{*}\PY{n}{RR}\PY{o}{*}\PY{n}{ABSs}\PY{o}{*}\PY{n}{ET}\PY{o}{*}\PY{n}{EF}\PY{o}{*}\PY{n}{ED}\PY{p}{)}\PY{o}{/}\PY{p}{(}\PY{n}{BW}\PY{o}{*}\PY{n}{AT}\PY{o}{*}\PY{l+m+mi}{365}\PY{p}{)}
    \PY{k}{return}\PY{p}{(}\PY{n}{CInh}\PY{p}{)}

\PY{c+c1}{\PYZsh{} populate values Child 1\PYZhy{}6}
\PY{n}{IR}\PY{o}{=} \PY{l+m+mf}{0.25} \PY{c+c1}{\PYZsh{}m\PYZca{}3/hr}
\PY{n}{RR}\PY{o}{=}\PY{l+m+mi}{100}\PY{o}{/}\PY{l+m+mi}{100} \PY{c+c1}{\PYZsh{}x/100\PYZpc{} }
\PY{n}{ABSs}\PY{o}{=}\PY{l+m+mi}{100}\PY{o}{/}\PY{l+m+mi}{100} \PY{c+c1}{\PYZsh{}x/100\PYZpc{}}
\PY{n}{ET}\PY{o}{=} \PY{l+m+mi}{12} \PY{c+c1}{\PYZsh{}hr/d}
\PY{n}{EF}\PY{o}{=} \PY{l+m+mi}{365} \PY{c+c1}{\PYZsh{}d/yr}
\PY{n}{ED}\PY{o}{=} \PY{l+m+mi}{5} \PY{c+c1}{\PYZsh{}yr}
\PY{n}{BW}\PY{o}{=} \PY{l+m+mi}{16} \PY{c+c1}{\PYZsh{}kg, child}
\PY{n}{AT}\PY{o}{=} \PY{l+m+mi}{70} \PY{c+c1}{\PYZsh{}yr}

\PY{n}{childCInh}\PY{o}{=}\PY{n}{CInh}\PY{p}{(}\PY{n}{CA}\PY{p}{,}\PY{n}{IR}\PY{p}{,}\PY{n}{RR}\PY{p}{,}\PY{n}{ABSs}\PY{p}{,}\PY{n}{ET}\PY{p}{,}\PY{n}{EF}\PY{p}{,}\PY{n}{ED}\PY{p}{,}\PY{n}{BW}\PY{p}{,}\PY{n}{AT}\PY{p}{)}
\PY{n+nb}{print}\PY{p}{(}\PY{l+s+s2}{\PYZdq{}}\PY{l+s+s2}{CInh for Child 1\PYZhy{}6 y.o.}\PY{l+s+s2}{\PYZdq{}}\PY{p}{,}\PY{n+nb}{round}\PY{p}{(}\PY{n}{childCInh}\PY{p}{,}\PY{l+m+mi}{9}\PY{p}{)}\PY{p}{)}

\PY{c+c1}{\PYZsh{} populate values Adult 19\PYZhy{}70y.o.}
\PY{n}{CA}\PY{o}{=}\PY{l+m+mf}{5.4e\PYZhy{}04} \PY{c+c1}{\PYZsh{} mg/m\PYZca{}3}
\PY{n}{IR}\PY{o}{=} \PY{l+m+mf}{0.83} \PY{c+c1}{\PYZsh{}m\PYZca{}3/hr}
\PY{n}{RR}\PY{o}{=}\PY{l+m+mi}{100}\PY{o}{/}\PY{l+m+mi}{100} \PY{c+c1}{\PYZsh{}x/100\PYZpc{} }
\PY{n}{ABSs}\PY{o}{=}\PY{l+m+mi}{100}\PY{o}{/}\PY{l+m+mi}{100} \PY{c+c1}{\PYZsh{}x/100\PYZpc{}}
\PY{n}{ET}\PY{o}{=} \PY{l+m+mi}{12} \PY{c+c1}{\PYZsh{}hr/d}
\PY{n}{EF}\PY{o}{=} \PY{l+m+mi}{365} \PY{c+c1}{\PYZsh{}d/yr}
\PY{n}{ED}\PY{o}{=} \PY{l+m+mi}{58} \PY{c+c1}{\PYZsh{}yr}
\PY{n}{BW}\PY{o}{=} \PY{l+m+mi}{70} \PY{c+c1}{\PYZsh{}kg, adolt}
\PY{n}{AT}\PY{o}{=} \PY{l+m+mi}{70} \PY{c+c1}{\PYZsh{}yr}

\PY{n}{adoltCInh}\PY{o}{=}\PY{n}{CInh}\PY{p}{(}\PY{n}{CA}\PY{p}{,}\PY{n}{IR}\PY{p}{,}\PY{n}{RR}\PY{p}{,}\PY{n}{ABSs}\PY{p}{,}\PY{n}{ET}\PY{p}{,}\PY{n}{EF}\PY{p}{,}\PY{n}{ED}\PY{p}{,}\PY{n}{BW}\PY{p}{,}\PY{n}{AT}\PY{p}{)}
\PY{n+nb}{print}\PY{p}{(}\PY{l+s+s2}{\PYZdq{}}\PY{l+s+s2}{CInh for Adolt }\PY{l+s+s2}{\PYZdq{}}\PY{p}{,}\PY{n+nb}{round}\PY{p}{(}\PY{n}{adoltCInh}\PY{p}{,}\PY{l+m+mi}{9}\PY{p}{)}\PY{p}{)}

\PY{c+c1}{\PYZsh{} use formula for CR from readings}

\PY{k}{def} \PY{n+nf}{CR}\PY{p}{(}\PY{n}{SF}\PY{p}{,}\PY{n}{Cinh}\PY{p}{)}\PY{p}{:}
    \PY{n}{CR}\PY{o}{=}\PY{n}{SF}\PY{o}{*}\PY{n}{Cinh}
    \PY{k}{return}\PY{p}{(}\PY{n}{CR}\PY{p}{)}

\PY{n}{childCR}\PY{o}{=}\PY{n}{CR}\PY{p}{(}\PY{n}{SF}\PY{p}{,}\PY{n}{childCInh}\PY{p}{)}
\PY{n}{adoltCR}\PY{o}{=}\PY{n}{CR}\PY{p}{(}\PY{n}{SF}\PY{p}{,}\PY{n}{adoltCInh}\PY{p}{)}
\PY{n+nb}{print}\PY{p}{(}\PY{l+s+s2}{\PYZdq{}}\PY{l+s+s2}{CR for Child }\PY{l+s+s2}{\PYZdq{}}\PY{p}{,}\PY{n+nb}{round}\PY{p}{(}\PY{n}{childCR}\PY{p}{,}\PY{l+m+mi}{9}\PY{p}{)}\PY{p}{)}
\PY{n+nb}{print}\PY{p}{(}\PY{l+s+s2}{\PYZdq{}}\PY{l+s+s2}{CR for Adolt }\PY{l+s+s2}{\PYZdq{}}\PY{p}{,}\PY{n+nb}{round}\PY{p}{(}\PY{n}{adoltCR}\PY{p}{,}\PY{l+m+mi}{9}\PY{p}{)}\PY{p}{)}
\end{Verbatim}
\end{tcolorbox}

    \begin{Verbatim}[commandchars=\\\{\}]
CInh for Child 1-6 y.o. 7.232e-06
CInh for Adolt  6.3663e-05
CR for Child  4.4116e-05
CR for Adolt  0.000388342
    \end{Verbatim}

    \hypertarget{problem-2}{%
\subsection{Problem 2}\label{problem-2}}

The same site also caused off-site lead concentrations that can cause
non-cancer effects on the residents. The RfD for lead is 6.90x10-4
\((\frac{mg}{kg d})^{-1}\). We will consider dermal exposures in this
problem, with a lead concentration of \(260~\frac{mg}{kg}\) in the soil,
and an absorption factor of 10 percent for the young children and 5
percent for adults.

Determine:

\begin{enumerate}
\def\labelenumi{\arabic{enumi}.}
\tightlist
\item
  The NCDEX for residents that are children 1-6 years of age and adults.
\item
  The hazard quotients due to these NCDEX values for the children and
  adults.
\end{enumerate}

Show all calculations and identify all parameter values used.

    \begin{tcolorbox}[breakable, size=fbox, boxrule=1pt, pad at break*=1mm,colback=cellbackground, colframe=cellborder]
\prompt{In}{incolor}{24}{\boxspacing}
\begin{Verbatim}[commandchars=\\\{\}]
\PY{c+c1}{\PYZsh{} Enter your solution below, or attach separate sheet(s) with your solution.}
\PY{c+c1}{\PYZsh{} given}
\PY{n}{RfD} \PY{o}{=} \PY{l+m+mf}{6.9e\PYZhy{}04} \PY{c+c1}{\PYZsh{}mg/kgd}
\PY{n}{CS} \PY{o}{=} \PY{l+m+mi}{260} \PY{c+c1}{\PYZsh{}mg/kg in soil}

\PY{c+c1}{\PYZsh{} use formula for NCDEX from readings}
\PY{k}{def} \PY{n+nf}{NCDEX}\PY{p}{(}\PY{n}{CS}\PY{p}{,}\PY{n}{CF}\PY{p}{,}\PY{n}{SA}\PY{p}{,}\PY{n}{AF}\PY{p}{,}\PY{n}{ABSs}\PY{p}{,}\PY{n}{SM}\PY{p}{,}\PY{n}{EF}\PY{p}{,}\PY{n}{ED}\PY{p}{,}\PY{n}{BW}\PY{p}{,}\PY{n}{AT}\PY{p}{,}\PY{n}{AvT}\PY{p}{)}\PY{p}{:}
    \PY{n}{NCDEX} \PY{o}{=} \PY{p}{(}\PY{n}{CS}\PY{o}{*}\PY{n}{CF}\PY{o}{*}\PY{n}{SA}\PY{o}{*}\PY{n}{AF}\PY{o}{*}\PY{n}{AT}\PY{o}{*}\PY{n}{ABSs}\PY{o}{*}\PY{n}{SM}\PY{o}{*}\PY{n}{EF}\PY{o}{*}\PY{n}{ED}\PY{p}{)}\PY{o}{/}\PY{p}{(}\PY{n}{BW}\PY{o}{*}\PY{n}{AvT}\PY{o}{*}\PY{l+m+mi}{365}\PY{p}{)}
    \PY{k}{return}\PY{p}{(}\PY{n}{NCDEX}\PY{p}{)}

\PY{c+c1}{\PYZsh{} Look up inputs}
\PY{n}{CF} \PY{o}{=} \PY{l+m+mf}{1e\PYZhy{}06} \PY{c+c1}{\PYZsh{}kg/mg}
\PY{n}{SA} \PY{o}{=} \PY{l+m+mi}{6980} \PY{c+c1}{\PYZsh{}cm\PYZca{}2/d}
\PY{n}{AT} \PY{o}{=} \PY{l+m+mi}{20}\PY{o}{/}\PY{l+m+mi}{100} \PY{c+c1}{\PYZsh{}\PYZpc{}\PYZhy{}child}
\PY{n}{AF} \PY{o}{=} \PY{l+m+mf}{0.75} \PY{c+c1}{\PYZsh{}mg/cm\PYZca{}2}
\PY{n}{ABSs} \PY{o}{=} \PY{l+m+mi}{10}\PY{o}{/}\PY{l+m+mi}{100} \PY{c+c1}{\PYZsh{}\PYZpc{}\PYZhy{}child}
\PY{n}{SM} \PY{o}{=} \PY{l+m+mf}{0.15}
\PY{n}{EF} \PY{o}{=} \PY{l+m+mi}{330} \PY{c+c1}{\PYZsh{}d/yr}
\PY{n}{ED} \PY{o}{=} \PY{l+m+mi}{5} \PY{c+c1}{\PYZsh{}yr child}
\PY{n}{BW} \PY{o}{=} \PY{l+m+mi}{16} \PY{c+c1}{\PYZsh{}kg child}
\PY{n}{AvT} \PY{o}{=} \PY{l+m+mi}{5} \PY{c+c1}{\PYZsh{}yr child}

\PY{n}{childNCDEX}\PY{o}{=} \PY{n}{NCDEX}\PY{p}{(}\PY{n}{CS}\PY{p}{,}\PY{n}{CF}\PY{p}{,}\PY{n}{SA}\PY{p}{,}\PY{n}{AF}\PY{p}{,}\PY{n}{ABSs}\PY{p}{,}\PY{n}{SM}\PY{p}{,}\PY{n}{EF}\PY{p}{,}\PY{n}{ED}\PY{p}{,}\PY{n}{BW}\PY{p}{,}\PY{n}{AT}\PY{p}{,}\PY{n}{AvT}\PY{p}{)}
\PY{n+nb}{print}\PY{p}{(}\PY{l+s+s2}{\PYZdq{}}\PY{l+s+s2}{NCEDX child}\PY{l+s+s2}{\PYZdq{}}\PY{p}{,}\PY{n+nb}{round}\PY{p}{(}\PY{n}{childNCDEX}\PY{p}{,}\PY{l+m+mi}{6}\PY{p}{)}\PY{p}{)}

\PY{c+c1}{\PYZsh{} repeat fro adult}

\PY{n}{SA} \PY{o}{=} \PY{l+m+mi}{18150} \PY{c+c1}{\PYZsh{}cm\PYZca{}2/d}
\PY{n}{AT} \PY{o}{=} \PY{l+m+mi}{10}\PY{o}{/}\PY{l+m+mi}{100} \PY{c+c1}{\PYZsh{}\PYZpc{}\PYZhy{}adult}
\PY{n}{ABSs} \PY{o}{=} \PY{l+m+mi}{5}\PY{o}{/}\PY{l+m+mi}{100} \PY{c+c1}{\PYZsh{}\PYZpc{}\PYZhy{}adult}
\PY{n}{ED} \PY{o}{=} \PY{l+m+mi}{58} \PY{c+c1}{\PYZsh{}yr DULT}
\PY{n}{BW} \PY{o}{=} \PY{l+m+mi}{70} \PY{c+c1}{\PYZsh{}kg adult}
\PY{n}{AvT} \PY{o}{=} \PY{l+m+mi}{58} \PY{c+c1}{\PYZsh{}yr adult}

\PY{n}{adultNCDEX}\PY{o}{=} \PY{n}{NCDEX}\PY{p}{(}\PY{n}{CS}\PY{p}{,}\PY{n}{CF}\PY{p}{,}\PY{n}{SA}\PY{p}{,}\PY{n}{AF}\PY{p}{,}\PY{n}{ABSs}\PY{p}{,}\PY{n}{SM}\PY{p}{,}\PY{n}{EF}\PY{p}{,}\PY{n}{ED}\PY{p}{,}\PY{n}{BW}\PY{p}{,}\PY{n}{AT}\PY{p}{,}\PY{n}{AvT}\PY{p}{)}
\PY{n+nb}{print}\PY{p}{(}\PY{l+s+s2}{\PYZdq{}}\PY{l+s+s2}{NCEDX adolt}\PY{l+s+s2}{\PYZdq{}}\PY{p}{,}\PY{n+nb}{round}\PY{p}{(}\PY{n}{adultNCDEX}\PY{p}{,}\PY{l+m+mi}{6}\PY{p}{)}\PY{p}{)}

\PY{c+c1}{\PYZsh{} use hazard quotiet formula}

\PY{k}{def} \PY{n+nf}{HQ}\PY{p}{(}\PY{n}{E}\PY{p}{,}\PY{n}{RfD}\PY{p}{)}\PY{p}{:}
    \PY{n}{HQ} \PY{o}{=} \PY{n}{E}\PY{o}{/}\PY{n}{RfD}
    \PY{k}{return}\PY{p}{(}\PY{n}{HQ}\PY{p}{)}

\PY{n}{childHQ} \PY{o}{=} \PY{n}{HQ}\PY{p}{(}\PY{n}{childNCDEX}\PY{p}{,}\PY{n}{RfD}\PY{p}{)}
\PY{n}{adultHQ} \PY{o}{=} \PY{n}{HQ}\PY{p}{(}\PY{n}{adultNCDEX}\PY{p}{,}\PY{n}{RfD}\PY{p}{)}

\PY{n+nb}{print}\PY{p}{(}\PY{l+s+s2}{\PYZdq{}}\PY{l+s+s2}{HQ child}\PY{l+s+s2}{\PYZdq{}}\PY{p}{,}\PY{n+nb}{round}\PY{p}{(}\PY{n}{childHQ}\PY{p}{,}\PY{l+m+mi}{3}\PY{p}{)}\PY{p}{)}
\PY{n+nb}{print}\PY{p}{(}\PY{l+s+s2}{\PYZdq{}}\PY{l+s+s2}{HQ adult}\PY{l+s+s2}{\PYZdq{}}\PY{p}{,}\PY{n+nb}{round}\PY{p}{(}\PY{n}{adultHQ}\PY{p}{,}\PY{l+m+mi}{3}\PY{p}{)}\PY{p}{)}
\end{Verbatim}
\end{tcolorbox}

    \begin{Verbatim}[commandchars=\\\{\}]
NCEDX child 0.000231
NCEDX adolt 3.4e-05
HQ child 0.334
HQ adult 0.05
    \end{Verbatim}

    \hypertarget{problem-3}{%
\subsection{Problem 3}\label{problem-3}}

A contaminated groundwater that is a potential drinking water source has
a manganese concentration of \(0.36~\frac{mg}{L}\). The RfD for
manganese is \(0.10~\frac{mg}{kg \cdot d}\). We will consider effects on
children 6-12 (drinking 1 L/d) and adults (2 L/d).

Determine: 1. The NCIng for children 6-12 and adults drinking this
water. 2. The hazard quotients due to these NCIng values for the
children and adults.

Show all calculations and identify all parameter values used.

    \begin{tcolorbox}[breakable, size=fbox, boxrule=1pt, pad at break*=1mm,colback=cellbackground, colframe=cellborder]
\prompt{In}{incolor}{29}{\boxspacing}
\begin{Verbatim}[commandchars=\\\{\}]
\PY{c+c1}{\PYZsh{} Enter your solution below, or attach separate sheet(s) with your solution.}
\PY{c+c1}{\PYZsh{} given}
\PY{n}{CW} \PY{o}{=} \PY{l+m+mf}{0.36} \PY{c+c1}{\PYZsh{}mg/L}
\PY{n}{RfD} \PY{o}{=} \PY{l+m+mf}{0.10} \PY{c+c1}{\PYZsh{}mg/kgd}
\PY{n}{IRc} \PY{o}{=} \PY{l+m+mi}{1} \PY{c+c1}{\PYZsh{}L/d}
\PY{n}{IRa} \PY{o}{=} \PY{l+m+mi}{2} \PY{c+c1}{\PYZsh{}L/d}

\PY{c+c1}{\PYZsh{} use formula for NCing from readings}
\PY{k}{def} \PY{n+nf}{NCing}\PY{p}{(}\PY{n}{CW}\PY{p}{,}\PY{n}{IR}\PY{p}{,}\PY{n}{FI}\PY{p}{,}\PY{n}{ABSs}\PY{p}{,}\PY{n}{EF}\PY{p}{,}\PY{n}{ED}\PY{p}{,}\PY{n}{BW}\PY{p}{,}\PY{n}{AT}\PY{p}{)}\PY{p}{:}
    \PY{n}{NCing}\PY{o}{=}\PY{p}{(}\PY{n}{CW}\PY{o}{*}\PY{n}{IR}\PY{o}{*}\PY{n}{FI}\PY{o}{*}\PY{n}{ABSs}\PY{o}{*}\PY{n}{EF}\PY{o}{*}\PY{n}{ED}\PY{p}{)}\PY{o}{/}\PY{p}{(}\PY{n}{BW}\PY{o}{*}\PY{n}{AT}\PY{o}{*}\PY{l+m+mi}{365}\PY{p}{)}
    \PY{k}{return}\PY{p}{(}\PY{n}{NCing}\PY{p}{)}

\PY{c+c1}{\PYZsh{} populate input values}
\PY{n}{FI} \PY{o}{=} \PY{l+m+mi}{100}\PY{o}{/}\PY{l+m+mi}{100} \PY{c+c1}{\PYZsh{}x/100 \PYZpc{}}
\PY{n}{ABSs} \PY{o}{=} \PY{l+m+mi}{100}\PY{o}{/}\PY{l+m+mi}{100} \PY{c+c1}{\PYZsh{}x/100 \PYZpc{}}
\PY{n}{EF} \PY{o}{=} \PY{l+m+mi}{365} \PY{c+c1}{\PYZsh{}d/yr}
\PY{n}{EDc} \PY{o}{=} \PY{l+m+mi}{6} \PY{c+c1}{\PYZsh{}yr}
\PY{n}{EDa} \PY{o}{=} \PY{l+m+mi}{58} \PY{c+c1}{\PYZsh{}yr}
\PY{n}{BWc} \PY{o}{=} \PY{l+m+mi}{29} \PY{c+c1}{\PYZsh{}kg}
\PY{n}{BWa} \PY{o}{=} \PY{l+m+mi}{70} \PY{c+c1}{\PYZsh{}kg}
\PY{n}{ATc} \PY{o}{=} \PY{l+m+mi}{6} \PY{c+c1}{\PYZsh{}yr}
\PY{n}{ATa} \PY{o}{=} \PY{l+m+mi}{58} \PY{c+c1}{\PYZsh{}yr}

\PY{n}{childNCing}\PY{o}{=}\PY{n}{NCing}\PY{p}{(}\PY{n}{CW}\PY{p}{,}\PY{n}{IRc}\PY{p}{,}\PY{n}{FI}\PY{p}{,}\PY{n}{ABSs}\PY{p}{,}\PY{n}{EF}\PY{p}{,}\PY{n}{EDc}\PY{p}{,}\PY{n}{BWc}\PY{p}{,}\PY{n}{ATc}\PY{p}{)}
\PY{n}{adultNCing}\PY{o}{=}\PY{n}{NCing}\PY{p}{(}\PY{n}{CW}\PY{p}{,}\PY{n}{IRa}\PY{p}{,}\PY{n}{FI}\PY{p}{,}\PY{n}{ABSs}\PY{p}{,}\PY{n}{EF}\PY{p}{,}\PY{n}{EDa}\PY{p}{,}\PY{n}{BWa}\PY{p}{,}\PY{n}{ATa}\PY{p}{)}
\PY{n}{childHQ} \PY{o}{=} \PY{n}{HQ}\PY{p}{(}\PY{n}{childNCing}\PY{p}{,}\PY{n}{RfD}\PY{p}{)}
\PY{n}{adultHQ} \PY{o}{=} \PY{n}{HQ}\PY{p}{(}\PY{n}{adultNCing}\PY{p}{,}\PY{n}{RfD}\PY{p}{)}

\PY{n+nb}{print}\PY{p}{(}\PY{l+s+s2}{\PYZdq{}}\PY{l+s+s2}{NCing child}\PY{l+s+s2}{\PYZdq{}}\PY{p}{,}\PY{n+nb}{round}\PY{p}{(}\PY{n}{childNCing}\PY{p}{,}\PY{l+m+mi}{6}\PY{p}{)}\PY{p}{)}
\PY{n+nb}{print}\PY{p}{(}\PY{l+s+s2}{\PYZdq{}}\PY{l+s+s2}{NCing adolt}\PY{l+s+s2}{\PYZdq{}}\PY{p}{,}\PY{n+nb}{round}\PY{p}{(}\PY{n}{adultNCing}\PY{p}{,}\PY{l+m+mi}{6}\PY{p}{)}\PY{p}{)}
\PY{n+nb}{print}\PY{p}{(}\PY{l+s+s2}{\PYZdq{}}\PY{l+s+s2}{HQ child}\PY{l+s+s2}{\PYZdq{}}\PY{p}{,}\PY{n+nb}{round}\PY{p}{(}\PY{n}{childHQ}\PY{p}{,}\PY{l+m+mi}{3}\PY{p}{)}\PY{p}{)}
\PY{n+nb}{print}\PY{p}{(}\PY{l+s+s2}{\PYZdq{}}\PY{l+s+s2}{HQ adult}\PY{l+s+s2}{\PYZdq{}}\PY{p}{,}\PY{n+nb}{round}\PY{p}{(}\PY{n}{adultHQ}\PY{p}{,}\PY{l+m+mi}{3}\PY{p}{)}\PY{p}{)}
\end{Verbatim}
\end{tcolorbox}

    \begin{Verbatim}[commandchars=\\\{\}]
NCing child 0.012414
NCing adolt 0.010286
HQ child 0.124
HQ adult 0.103
    \end{Verbatim}

    \hypertarget{problem-4}{%
\subsection{Problem 4}\label{problem-4}}

An animal exposure study was performed to determine an acceptable
drinking water concentration for a chemical that causes liver disease in
rats and is assumed to have a nonzero threshold. The following results
were obtained.

\textbf{Control Group} Comparison to historical records: no evidence of
premature deaths Time of sacrifice: all surviving rats were sacrificed
at 18 months Initial number: 100 Number of rats with liver disease: 3

\textbf{Test Group} Exposure conditions (lowest observed effect): 140
mg/L, 30 mL/d for a median of 12 months Time of sacrifice: all surviving
rats were sacrificed at 18 months Comparison of weight and survival
curves: no differences between test and control rats Median adult
weight: 0.4 kg Initial number exposed: 100 Number of rats with liver
disease: 12

Determine:

\begin{enumerate}
\def\labelenumi{\arabic{enumi}.}
\tightlist
\item
  The LOAEL for the rats based on this study.
\item
  The RfD for humans by adjusting for uncertainty. This result is
  subchronic animal data with no human exposure data available.
\item
  Convert the RfD to an acceptable drinking water concentration.
\end{enumerate}

    \begin{tcolorbox}[breakable, size=fbox, boxrule=1pt, pad at break*=1mm,colback=cellbackground, colframe=cellborder]
\prompt{In}{incolor}{30}{\boxspacing}
\begin{Verbatim}[commandchars=\\\{\}]
\PY{c+c1}{\PYZsh{} Enter your solution below, or attach separate sheet(s) with your solution.}

\PY{c+c1}{\PYZsh{}1.The LOAEL for the rats based on this study:}
\PY{n}{CW}\PY{o}{=}\PY{l+m+mi}{140} \PY{c+c1}{\PYZsh{} mg/l}
\PY{n}{IR}\PY{o}{=}\PY{l+m+mf}{0.030} \PY{c+c1}{\PYZsh{} L/d}
\PY{n}{FI}\PY{o}{=}\PY{l+m+mi}{1}
\PY{n}{ABSs}\PY{o}{=}\PY{l+m+mi}{1}
\PY{n}{EF}\PY{o}{=}\PY{l+m+mi}{365} \PY{c+c1}{\PYZsh{} d/yr}
\PY{n}{ED}\PY{o}{=} \PY{l+m+mi}{1} \PY{c+c1}{\PYZsh{} yr}
\PY{n}{BW}\PY{o}{=}\PY{l+m+mf}{0.4} \PY{c+c1}{\PYZsh{} kg}
\PY{n}{AT}\PY{o}{=} \PY{l+m+mi}{1} \PY{c+c1}{\PYZsh{} yr}
\PY{n}{NCIng\PYZus{}Rat}\PY{o}{=}\PY{p}{(}\PY{n}{CW}\PY{o}{*}\PY{n}{IR}\PY{o}{*}\PY{n}{FI}\PY{o}{*}\PY{n}{ABSs}\PY{o}{*}\PY{n}{EF}\PY{o}{*}\PY{n}{ED}\PY{p}{)}\PY{o}{/}\PY{p}{(}\PY{n}{BW}\PY{o}{*}\PY{n}{AT}\PY{o}{*}\PY{l+m+mi}{365}\PY{p}{)}
\PY{n+nb}{print}\PY{p}{(}\PY{l+s+s2}{\PYZdq{}}\PY{l+s+s2}{The LOAEL for the rats = }\PY{l+s+si}{\PYZpc{}0.3f}\PY{l+s+s2}{ mg/(kgd)}\PY{l+s+s2}{\PYZdq{}} \PY{o}{\PYZpc{}}\PY{k}{NCIng\PYZus{}Rat}) 
\end{Verbatim}
\end{tcolorbox}

    \begin{Verbatim}[commandchars=\\\{\}]
The LOAEL for the rats = 10.500 mg/(kgd)
    \end{Verbatim}

    \begin{tcolorbox}[breakable, size=fbox, boxrule=1pt, pad at break*=1mm,colback=cellbackground, colframe=cellborder]
\prompt{In}{incolor}{32}{\boxspacing}
\begin{Verbatim}[commandchars=\\\{\}]
\PY{c+c1}{\PYZsh{}2.The RfD for humans by adjusting for uncertainty:}
\PY{n}{UF}\PY{o}{=}\PY{l+m+mi}{10}\PY{o}{*}\PY{o}{*}\PY{p}{(}\PY{l+m+mi}{4}\PY{p}{)} \PY{c+c1}{\PYZsh{} UF=10H*10A*10S*10L=10\PYZca{}4}
\PY{n}{MF}\PY{o}{=}\PY{l+m+mi}{1}
\PY{n}{RfD}\PY{o}{=}\PY{n}{NCIng\PYZus{}Rat}\PY{o}{/}\PY{p}{(}\PY{n}{UF}\PY{o}{*}\PY{n}{MF}\PY{p}{)}
\PY{n+nb}{print}\PY{p}{(}\PY{l+s+s2}{\PYZdq{}}\PY{l+s+s2}{ RfD for humans by adjusting for uncertainty = }\PY{l+s+si}{\PYZpc{}0.7f}\PY{l+s+s2}{ mg/(kgd)}\PY{l+s+s2}{\PYZdq{}} \PY{o}{\PYZpc{}}\PY{k}{RfD})
\end{Verbatim}
\end{tcolorbox}

    \begin{Verbatim}[commandchars=\\\{\}]
 RfD for humans by adjusting for uncertainty = 0.0010500 mg/(kgd)
    \end{Verbatim}

    \begin{tcolorbox}[breakable, size=fbox, boxrule=1pt, pad at break*=1mm,colback=cellbackground, colframe=cellborder]
\prompt{In}{incolor}{33}{\boxspacing}
\begin{Verbatim}[commandchars=\\\{\}]
\PY{c+c1}{\PYZsh{}3.Converting the RfD to an acceptable drinking water concentration:}
\PY{n}{BW}\PY{o}{=}\PY{l+m+mi}{70} \PY{c+c1}{\PYZsh{} kg}
\PY{n}{IR}\PY{o}{=}\PY{l+m+mi}{2} \PY{c+c1}{\PYZsh{} L/d}
\PY{n}{DWEL}\PY{o}{=}\PY{p}{(}\PY{n}{RfD}\PY{o}{*}\PY{n}{BW}\PY{p}{)}\PY{o}{/}\PY{n}{IR}
\PY{n+nb}{print}\PY{p}{(}\PY{l+s+s2}{\PYZdq{}}\PY{l+s+s2}{ RfD to an acceptable drinking water concentration = }\PY{l+s+si}{\PYZpc{}0.7f}\PY{l+s+s2}{ mg/l}\PY{l+s+s2}{\PYZdq{}} \PY{o}{\PYZpc{}}\PY{k}{DWEL})
\end{Verbatim}
\end{tcolorbox}

    \begin{Verbatim}[commandchars=\\\{\}]
 RfD to an acceptable drinking water concentration = 0.0367500 mg/l
    \end{Verbatim}

    \hypertarget{problem-5}{%
\subsection{Problem 5}\label{problem-5}}

Visit the EPA's IRIS system website
(http://www.epa.gov/iriswebp/iris/index.html)

Determine: 1. Your favorite toxic or carcinogenic substance and print
(or screen capture) the Quick View page for your choice.

    \begin{tcolorbox}[breakable, size=fbox, boxrule=1pt, pad at break*=1mm,colback=cellbackground, colframe=cellborder]
\prompt{In}{incolor}{ }{\boxspacing}
\begin{Verbatim}[commandchars=\\\{\}]

\end{Verbatim}
\end{tcolorbox}


    % Add a bibliography block to the postdoc
    
    
    
\end{document}
