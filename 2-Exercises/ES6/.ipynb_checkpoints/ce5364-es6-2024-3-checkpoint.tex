\documentclass[12pt]{article}
\usepackage{geometry}                % See geometry.pdf to learn the layout options. There are lots.
\geometry{letterpaper}                   % ... or a4paper or a5paper or ... 
%\geometry{landscape}                % Activate for for rotated page geometry
\usepackage[parfill]{parskip}    % Activate to begin paragraphs with an empty line rather than an indent
\usepackage{daves,fancyhdr,natbib,graphicx,dcolumn,amsmath,lastpage,url}
\usepackage{amsmath,amssymb,epstopdf,longtable}
\usepackage[final]{pdfpages}
\DeclareGraphicsRule{.tif}{png}{.png}{`convert #1 `dirname #1`/`basename #1 .tif`.png}
\pagestyle{fancy}
\lhead{CE 5364 -- Groundwater Transport Phenomena }
\rhead{FALL 2024}
\lfoot{ES6}
\cfoot{}
\rfoot{Page \thepage\ of \pageref{LastPage}}
\renewcommand\headrulewidth{0pt}



\begin{document}
\begin{center}
{\textbf{{ CE 5364 Groundwater Transport Phenomena } \\ {Exercise Set 6}}}
\end{center}

\section*{\small{Exercises}}
\begin{enumerate} %% Problem Counter

%%%%%%%%%%%%%%%%%%%%%%%%%%%%%%%%%%%%%%%%%%%%%%%%%%%%

\item Improper waste disposal practices at an industrial site resulted in contamination of the soil on site by cadmium, a known carcinogen with a slope factor of 6.10 $(\frac{mg}{kg d})^{-1}$. We will consider the risk to off-site residents due to inhalation of airborne soil particles that include the cadmium.  Based on monitoring data, the concentration of cadmium in the air off site is $5.4 \times 10^{-4}~\frac{mg}{m^3}$.

Determine:

\begin{enumerate} %% Deliverable Counter
\item CInh for residents that are children 1-6 years of age and adults.
\item The cancer risk due to these CInh values for the children and adults. 
\end{enumerate} %% Deliverable Counter

Show all calculations and identify all parameter values used.

%\clearpage

%%%%%%%%%%%%%%%%%%%%%%%%%%%%%%%%%%%%%%%%%%%%%%%%%%%%%%%%%%
\item The same site also caused off-site lead concentrations that can cause non-cancer effects on the residents. The RfD for lead is 6.90x10-4 $(\frac{mg}{kg d})^{-1}$.  We will consider dermal exposures in this problem, with a lead concentration of $260~\frac{mg}{kg}$ in the soil, and an absorption factor of 10 percent for the young children and 5 percent for adults. 

Determine:

\begin{enumerate} %% Deliverable Counter
\item The NCDEX for residents that are children 1-6 years of age and adults. 
\item The hazard quotients due to these NCDEX values for the children and adults.
\end{enumerate} %% Deliverable Counter

Show all calculations and identify all parameter values used. 

%\clearpage
%%%%%%%%%%%%%%%%%%%%%%%%%%%%%%%%%%%%%%%%%%%%%%%%%%%%%%%%%%

\item A contaminated groundwater that is a potential drinking water source has a manganese concentration of $0.36~\frac{mg}{L}$. The RfD for manganese is $0.10~\frac{mg}{kg \cdot d}$. We will consider effects on children 6-12 (drinking 1 L/d) and adults (2 L/d). 

Determine:
\begin{enumerate} %% Deliverable Counter
\item The NCIng for children 6-12 and adults drinking this water.
\item The hazard quotients due to these NCIng values for the children and adults. 
\end{enumerate} %% Deliverable Counter

Show all calculations and identify all parameter values used. 

\clearpage
%%%%%%%%%%%%%%%%%%%%%%%%%%%%%%%%%%%%%%%%%%%%%%%%%%%%%%
%%%%%%%%%%%%%%%%%%%%%%%%%%%%%%%%%%%%%%%%%%%%%%%%%%%%%%
\item An animal exposure study was performed to determine an acceptable drinking water concentration for a chemical that causes liver disease in rats and is assumed to have a nonzero threshold. The following results were obtained. 

\textbf{Control Group}\\
Comparison to historical records: no evidence of premature deaths 
Time of sacrifice: all surviving rats were sacrificed at 18 months 
Initial number: 100 
Number of rats with liver disease: 3 

\textbf{Test Group}
Exposure conditions (lowest observed effect): 140 mg/L, 30 mL/d for a median of 12 months 
Time of sacrifice: all surviving rats were sacrificed at 18 months 
Comparison of weight and survival curves: no differences between test and control rats 
Median adult weight: 0.4 kg 
Initial number exposed: 100 
Number of rats with liver disease: 12 

Determine:

\begin{enumerate} %% Deliverable Counter
\item The LOAEL for the rats based on this study.
\item The RfD for humans by adjusting for uncertainty. This result is subchronic animal data with no human exposure data available. 
\item Convert the RfD to an acceptable drinking water concentration. 
\end{enumerate} %% Deliverable Counter
%\clearpage
%%%%%%%%%%%%%%%%%%%%%%%%%%%%%%%%%%%%%%%%%%%%%%%%%%%%%%
\item Visit the EPA’s IRIS system website (http://www.epa.gov/iriswebp/iris/index.html)

Determine:
\begin{enumerate} %% Deliverable Counter
\item Your favorite toxic or carcinogenic substance and print (or screen capture) the Quick View page for your choice.
\end{enumerate} %% Deliverable Counter

%%%%%%%%%%%%%%%%%%%%%%%%%%%%%%%%%%%%%%%%%%%%%%%%%%%%%%%%%%%%%%%%%%%
\end{enumerate}%% Problem Counter

\end{document}  